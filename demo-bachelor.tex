%%
%% File encoding: UTF-8
%% 这是 `demo-bachelor.tex' 文件
%%
\documentclass[bachelor,ncu,newenv]{ncuthesis}
\usepackage[colorlinks,
            linkcolor=black,   % 普通链接颜色
            citecolor=black,   % 文献引用颜色 
            urlcolor=blue]{hyperref}
\ncusetup{
  info = {
    secret-level     = {公开},
    studentid        = {402200220047},
    title            = {南昌大学学位论文模板使用演示},
    title*           = {Demonstration of Nanchang University\\Thesis Template},
    author           = {朱金宝},
    department       = {物理},
    supervisor       = {舒富文},
    start-date       = {23},
    stop-date        = {27},
    college          = {物理与材料学院},
    major            = {物理学},
    major-class      = {物理学 1 班},
    supervisor-title = {教授},
    start-stop-date  = {2023 年 9 月—2027 年 6 月}
  }
}
\begin{document}
\maketitle
\frontmatter
\makedecaut
\phantomsection
\begin{abstract}
这里写摘要。
\keywords{关键词 1, 关键词 2, 关键词 3}
\end{abstract}
\phantomsection
\begin{abstract*}
This is abstract.
\keywords{kw1, kw2, kw3}
\end{abstract*}
\tableofcontents
\mainmatter
\chapter{引言}
\section{历史}
我们知道 lshort\cite{lshortcn} 是入门 \LaTeX 的必读文献。\footnote{脚注。}
\begin{figure}[htb]
\centering
\begin{subfigure}{.45\textwidth}
\centering
\includegraphics[width=5cm]{example-image}
\subcaption{这是子图}
\label{fig:subfigexp-a}
\end{subfigure}
\begin{subfigure}{.45\textwidth}
\centering
\includegraphics[width=5cm]{example-image}
\subcaption{这又是子图}
\label{fig:subfigexp-b}
\end{subfigure}
\begin{subfigure}{.45\textwidth}
\centering
\includegraphics[width=5cm]{example-image}
\subcaption{这还是子图}
\label{fig:subfigexp-c}
\end{subfigure}
\caption{子图排版示例}
\label{fig:subfig}
\end{figure}
\begin{law}[黑洞面积定律]
黑洞事件视界的面积不可能随着时间的推移而减小。
\end{law}
\begin{proof}
即得易见平凡,仿照上例显然。留作习题答案略,读者自证不难。
反之亦然同理,推论自然成立。略去过程 QED,由上可知证毕。
\end{proof}
\appendix
\chapter{公式推导}
\section{场方程}
附录内容。
\backmatter
\bibliography{references.bib}
\begin{acknowledgements}
感谢……
\signoff{朱金宝}{2022 年 12 月 31 日}
\end{acknowledgements}
\end{document}
%% 结束
%%