%%
%% File encoding: UTF-8
%% 这是 `demo-professional.tex' 文件
%%
\documentclass[professional,final,newenv]{ncuthesis}
%%
\ncusetup{
  % style = {
  %  footnote-num = plain,
  %  font         = times,
  %  cjk-font     = fandol
  % },
  info = {
    CLC             = {O}, % 可省略花括号,但最好不要
    secret-level*   = {公开},
    UDC             = {123.123.1},
    studentid       = {402200220047},
    degree          = {硕士},
    degree-type     = {种类},
    title           = {南昌大学学位论文模板使用演示},
    title*          = {Demonstration\\of\\Nanchang University Thesis Template},
    author          = {朱金宝},
    department      = {物理},
    supervisor      = {舒富文},
    field           = {名称},
    defence-date    = {2023 年 2 月 16 日},
    tdc-chairman    = {主席},
    reviewer        = {评阅人 1, 评阅人 2},
    completion-date = {2023 年 2 月 16 日}
  }
}
%%
% \usepackage{hyperref}
%%
\begin{document}
\maketitle
%%
\frontmatter
\makedecaut
%%
\begin{abstract}
这是摘要内容。
\keywords{关键词 1, 关键词 2, 关键词 3}
\end{abstract}
\begin{abstract*}
This is abstract.
\keywords{kw1, kw2, kw3}
\end{abstract*}
%%
\tableofcontents
%%
\mainmatter
\chapter{引言}
\section{历史}
我们知道 lshort\cite{lshortcn} 是入门 \LaTeX\ 的必读文献。\footnote{test}

\begin{figure}[htb]
\centering
\begin{subfigure}{.45\textwidth}
\centering
\includegraphics[width=5cm]{example-image}
\subcaption{这是子图}
\label{fig:subfigexp-a}
\end{subfigure}
\begin{subfigure}{.45\textwidth}
\centering
\includegraphics[width=5cm]{example-image}
\subcaption{这又是子图}
\label{fig:subfigexp-b}
\end{subfigure}
\begin{subfigure}{.45\textwidth}
\centering
\includegraphics[width=5cm]{example-image}
\subcaption{这还是子图}
\label{fig:subfigexp-c}
\end{subfigure}
\caption{子图排版示例}
\label{fig:subfig}
\end{figure}

\begin{law}[黑洞面积定律]
黑洞事件视界的面积不可能随着时间的推移而减小。
\end{law}
\begin{proof}
即得易见平凡,仿照上例显然。留作习题答案略,读者自证不难。反之亦然同理,推论自然成立。略去过程 QED,由上可知证毕。
\end{proof}
\appendix
\chapter{公式推导}
\section{Einstein 场方程}
%% 推导过程
%%
\backmatter
\bibliography{references.bib}
%%
\begin{acknowledgements}
感谢……
\signoff{朱金宝}{2022 年 12 月 31 日}
\end{acknowledgements}
%%
%%
% 注意,以下内容本科生毕业论文不能写,否则报错!
\begin{researchresults}
\begin{published}
\item 条目
\end{published}
\begin{tobepublished}
\item 条目
\end{tobepublished}
\begin{reports}
\item 条目
\end{reports}
\begin{others}[另外还有] % 标题作为可选参数
\item 条目
\end{others}
\begin{papers}
\item 条目
\end{papers}
\end{researchresults}
\end{document}
%% 结束
%%