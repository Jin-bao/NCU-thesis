%%
%% File encoding: UTF-8
%% 这是 `main.tex' 文件
%%
\documentclass[draft,newenv,newcmd]{ncuthesis}
%%
\ncusetup{
  info = {
    CLC             = {O},
    secret-level*   = {公开},
    UDC             = {},
    studentid       = {402200220047},
    degree          = {硕士},
    title           = {南昌大学学位论文模板使用演示},
    title*          = {Demonstration\\of\\Nanchang University Thesis Template},
    author          = {朱金宝},
    department      = {物理与材料学院},
    supervisor      = {舒富文教授},
    field           = {理学},
    major           = {天体物理},
    defence-date    = {2022 年 12 月 31 日},
    tdc-chairman    = {主席},
    reviewer        = {评阅人 1, 评阅人 2},
    completion-date = {2022 年 12 月 31 日},
    email           = {yyyyyyhdd@outlook.com},
    decautremark    = {这里填备注}
  }
}
%%
\begin{document}
\maketitle
%%
\frontmatter
%%
\makedecaut
%%
\begin{abstract}
这是摘要内容. 
\keywords{关键词 1, 关键词 2, 关键词 3}
\end{abstract}
%%
\begin{abstract*}
This is abstract.
\keywords{kw1, kw2, kw3}
\end{abstract*}
%%
\tableofcontents
%%
\mainmatter
%%
\chapter{引言}
\section{历史}
我们知道 lshort\cite{lshortcn} 是入门 \LaTeX\ 的必读文献.\footnote{这是脚注测试.}

\begin{theorem}
若满足条件
\[
  \sum_{j=0}^S q_j = 1,\quad
  \sum_{k\in\symbb{Z}} h_{k-2n}\bar{h}_k = \delta_{n0},\quad
  \sup \lvert Q(z)\rvert < 2^{N-1},
\]
则双尺度方程 $\varphi(x) = \sqrt{2}\sum_{k=0}^L h_k\varphi(2x-k)$ 迭代可解, 
且其解生成一个多分辨分析. 
\end{theorem}
\begin{proof}
即得易见平凡, 仿照上例显然. 留作习题答案略, 读者自证不难. 
反之亦然同理, 推论自然成立. 略去过程 QED, 由上可知证毕. \freeze
\end{proof}
\appendix
\chapter{公式推导}
\section{Einstein 场方程}
%% 推导过程
%%
\backmatter
%%
\bibliography{references.bib}
%%
\begin{acknowledgements}
感谢……
\signoff{朱金宝}{2022 年 12 月 31 日}
\end{acknowledgements}
%%
\begin{researchresults}
\begin{published}
\item OETIKER T, PARTL H, HYNA I, et al. 一份 (不太) 简短的
\LaTeXe\ 介绍 [EB/OL]. 2021.
\end{published}
\begin{tobepublished}
\item 即将发表内容
\item 马上发表内容
\end{tobepublished}
\begin{reports}
\item 研究报告
\end{reports}
\begin{others}[另外还有] % 标题作为可选参数
\item 其他的内容
\end{others}
不论如何, 还可以在 \textsf{researchresults} 环境内使用
\texttt{papers} 列表环境, 这个环境最多支持到第二级. 
\begin{papers}
\item 论文
\item 报告
\end{papers}
\end{researchresults}
\end{document}
%% 结束
%%