%%
%% File encoding: UTF-8
%% 这是 `main.tex' 文件
%%
%%
\documentclass[draft,newenv,newcmd]{ncuthesis}
%%
%%
\ncusetup{
  info = {
    CLC             = {},
    secret-level*   = {公开},
    UDC             = {},
    studentid       = {402200220047},
    degree          = {硕士},
    title           = {南昌大学学位论文模板使用演示},
    title*          = {Demonstration\\of\\Nanchang University Thesis Template},
    author          = {朱金宝},
    department      = {物理与材料学院},
    supervisor      = {舒富文\quad 教授},
    field           = {理学},
    major           = {天体物理},
    defence-date    = {2022 年 12 月 31 日},
    tdc-chairman    = {主席},
    reviewer        = {评阅人 1, 评阅人 2},
    completion-date = {2022 年 12 月 31 日},
    email           = {yyyyyyhdd@outlook.com},
    decautremark    = {}
  }
}
%%
%% 其他设置
% \usepackage{hyperref}
%%
%% 文档开始
\begin{document}
%%
%% 封面
\maketitle
%%
%% 前文
\frontmatter
%%
%% 声明和授权书
\makedecaut
%%
%% 中文摘要
\begin{abstract}
本文展示了 NCU-thesis 模板的使用效果. 用户可以根据示例文件代码, 仿照着写出自己的
毕业论文. 
\keywords{南昌大学, 拉泰赫模板, 使用演示}
\end{abstract}
%%
%% 英文摘要
\begin{abstract*}
This article demonstrates the use of the NCU-thesis \LaTeX\ template.
Users can write their own thesis based on the \verb|main.tex| file code.
\keywords{Nanchang University, \LaTeX\ template, demonstration}
\end{abstract*}
%%
%% 目录
\tableofcontents
%%
%% 正文
\mainmatter
%%
%%
\chapter{介绍}
\section{一些 \LaTeX 书籍推荐}
我们知道 lshort\cite{lshortcn} 是入门 \LaTeX 的必读文献.%
\footnote{参见 \texttt{https://github.com/CTeX-org/lshort-zh-cn}}
可以花上几个小时阅读一下, 能够避免被一些简单的问题耽误时间. 

除此之外, 还推荐刘海洋的《\LaTeX 入门》、胡伟的《\LaTeXe 完全学习手册》, 进阶
用户还可以看胡伟的《\LaTeXe 文类和宏包学习手册》. 
\section{图和表}
图和表可以看作宽和高较大的文字,\footnote{严格来说称为 ``盒子''.} 当前页空间不够
时, 图表自然就要去其他地方找位置.

\begin{figure}[htb]
\centering
\includegraphics[width=6cm]{example-image}
\caption{示例图片, 在 \texttt{draft} 模式下不显示图片内容, 只显示图片轮廓.}
\label{fig:example-fig}
\end{figure}

\begin{table}[htb]
\centering
\caption{把表注写在前面, 表注就会在表格上方.}
\begin{tabular}{lccr}
  \thickhline
  左对齐表头 & 居中对齐表头 & 居中对齐表头 & 右对齐表头 \\
  \hline
  一行一列   & 一行二列     & 一行三列     & 一行四列 \\
  二行一列   & 二行二列     & 二行三列     & 二行四列 \\
  三行一列   & 三行二列     & 三行三列     & 三行四列 \\
  四行一列   & 四行二列     & 四行三列     & 四行四列 \\
  五行一列   & 五行二列     & 五行三列     & 五行四列 \\
  六行一列   & 六行二列     & 六行三列     & 六行四列 \\
  七行一列   & 七行二列     & 七行三列     & 七行四列 \\
  八行一列   & 八行二列     & 八行三列     & 八行四列 \\
  \thickhline
\end{tabular}
\label{tab:example-tab}
\end{table}

由于 \verb|figure| 和 \verb|table| 环境是浮动的, 文本流有时会提前排版出来, 
比如表 \ref{tab:example-tab} 就排到了下一页. 

\section{数学公式}
对 Fourier 变换采用以下约定:
\begin{equation}
  (\symcal{F}\!f)(\xi) = \hat f(\xi) = \frac{1}{\sqrt{2\pi}}
    \int_{-\infty}^{\infty}\symup{e}^{-ix\xi}f(x)\,\symup{d}x.
\end{equation}
根据这一标准定义, 有
\[
  \lVert\hat f\rVert_{L^2} = \lVert f\rVert_{L^2},\quad
  \lvert\hat f(\xi)\rvert \leq (2\pi)^{-1/2}\lVert f\rVert_{L^1}.
\]
\begin{theorem}[Fubini 定理]
若 $\int\symup{d}x\,\lr[]{\int\symup{d}y\,\lvert f(x,y)\rvert} < \infty$, 则
\begin{equation}
  \int\symup{d}x\int\symup{d}y\,f(x,y)
    = \int\symup{d}x\,\lr[]{\int\symup{d}y\,f(x,y)}
    = \int\symup{d}y\,\lr[]{\int\symup{d}x\,f(x,y)},
\end{equation}
即积分顺序可以互换.
\end{theorem}
\begin{proof}
即得易见平凡, 仿照上例显然. 留作习题答案略, 读者自证不难. 
反之亦然同理, 推论自然成立. 略去过程 QED, 由上可知证毕. \freeze
\end{proof}
%%
%% 附录
\appendix
\chapter{附加材料}
\section{附录的章节标题}
附录与正文的章节标题用法其实并没有什么区别, 它们的效果也仅仅是编号方式不一样
而已. 
%%
%% 后文
\backmatter
%%
%% 参考文献
\bibliography{references.bib}
%%
%% 致谢
\begin{acknowledgements}
感谢我的朋友们!
\signoff{朱金宝}{2022 年 12 月 31 日}
\end{acknowledgements}
%%
%% 研究成果
\begin{researchresults}
\begin{published}
\item Oetiker T, Partl H, Hyna I, et al. 一份 (不太) 简短的
\LaTeXe\ 介绍 [EB/OL]. 2021.
\end{published}
\begin{tobepublished}
\item 待发表论文. 
\end{tobepublished}
\begin{reports}
\item 研究报告. 
\end{reports}
\begin{others}[另外还有]
\item 其他内容. 
\end{others}
不论如何, 还可以在 \textsf{researchresults} 环境内使用不带标题的
\texttt{papers} 列表环境, 这个环境最多支持到第二级. 
\begin{papers}
\item 论文.
\item 报告.
  \begin{papers}
    \item 专利. 
  \end{papers}
\end{papers}
\end{researchresults}
\end{document}
%% 结束
%%